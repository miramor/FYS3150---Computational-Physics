%%
%% Automatically generated file from DocOnce source
%% (https://github.com/hplgit/doconce/)
%%
%%


%-------------------- begin preamble ----------------------

\documentclass[%
oneside,                 % oneside: electronic viewing, twoside: printing
final,                   % draft: marks overfull hboxes, figures with paths
10pt]{article}

\listfiles               %  print all files needed to compile this document

\usepackage{relsize,makeidx,color,setspace,amsmath,amsfonts,amssymb}
\usepackage[table]{xcolor}
\usepackage{bm,ltablex,microtype}
\newcommand{\R}{\mathbb{R}}

\usepackage[pdftex]{graphicx}

\usepackage{fancyvrb} % packages needed for verbatim environments

\usepackage[T1]{fontenc}
%\usepackage[latin1]{inputenc}
\usepackage{ucs}
\usepackage[utf8x]{inputenc}

\usepackage{lmodern}         % Latin Modern fonts derived from Computer Modern

% Hyperlinks in PDF:
\definecolor{linkcolor}{rgb}{0,0,0.4}
\usepackage{hyperref}
\hypersetup{
    breaklinks=true,
    colorlinks=true,
    linkcolor=linkcolor,
    urlcolor=linkcolor,
    citecolor=black,
    filecolor=black,
    %filecolor=blue,
    pdfmenubar=true,
    pdftoolbar=true,
    bookmarksdepth=3   % Uncomment (and tweak) for PDF bookmarks with more levels than the TOC
    }
%\hyperbaseurl{}   % hyperlinks are relative to this root

\setcounter{tocdepth}{2}  % levels in table of contents

% --- fancyhdr package for fancy headers ---
\usepackage{fancyhdr}
\fancyhf{} % sets both header and footer to nothing
\renewcommand{\headrulewidth}{0pt}
\fancyfoot[LE,RO]{\thepage}
% Ensure copyright on titlepage (article style) and chapter pages (book style)
\fancypagestyle{plain}{
  \fancyhf{}
  \fancyfoot[C]{{\footnotesize \copyright\ 1999-2020, "Computational Physics I FYS3150/FYS4150":"http://www.uio.no/studier/emner/matnat/fys/FYS3150/index-eng.html". Released under CC Attribution-NonCommercial 4.0 license}}
%  \renewcommand{\footrulewidth}{0mm}
  \renewcommand{\headrulewidth}{0mm}
}
% Ensure copyright on titlepages with \thispagestyle{empty}
\fancypagestyle{empty}{
  \fancyhf{}
  \fancyfoot[C]{{\footnotesize \copyright\ 1999-2020, "Computational Physics I FYS3150/FYS4150":"http://www.uio.no/studier/emner/matnat/fys/FYS3150/index-eng.html". Released under CC Attribution-NonCommercial 4.0 license}}
  \renewcommand{\footrulewidth}{0mm}
  \renewcommand{\headrulewidth}{0mm}
}

\pagestyle{fancy}


% prevent orhpans and widows
\clubpenalty = 10000
\widowpenalty = 10000

% --- end of standard preamble for documents ---


% insert custom LaTeX commands...

\raggedbottom
\makeindex
\usepackage[totoc]{idxlayout}   % for index in the toc
\usepackage[nottoc]{tocbibind}  % for references/bibliography in the toc

%-------------------- end preamble ----------------------

\begin{document}

% matching end for #ifdef PREAMBLE

\newcommand{\exercisesection}[1]{\subsection*{#1}}


% ------------------- main content ----------------------



% ----------------- title -------------------------

\thispagestyle{empty}

\begin{center}
{\LARGE\bf
\begin{spacing}{1.25}
Project 1, deadline  September 9
\end{spacing}
}
\end{center}

% ----------------- author(s) -------------------------

\begin{center}
{\bf \href{{http://www.uio.no/studier/emner/matnat/fys/FYS3150/index-eng.html}}{Computational Physics I FYS3150/FYS4150}}
\end{center}

    \begin{center}
% List of all institutions:
\centerline{{\small Department of Physics, University of Oslo, Norway}}
\end{center}
    
% ----------------- end author(s) -------------------------

% --- begin date ---
\begin{center}
Aug 25, 2020
\end{center}
% --- end date ---

\vspace{1cm}

\subsection*{Abstract}
\subsection*{Introduction}
\subsection*{Method}
Given the one-dimensional Poisson equation with Dirichlet boundary conditions
\begin{equation*}
-u''(x) = f(x), \hspace{0.5cm} x\in(0,1), \hspace{0.5cm} u(0) = u(1) = 0.
\end{equation*}
We approximate the second derivative of $u$ with
\begin{equation*}
   -\frac{v_{i+1}+v_{i-1}-2v_i}{h^2} = f_i  \hspace{0.5cm} \mathrm{for} \hspace{0.1cm} i=1,\dots, n,
\end{equation*}
where $f_i=f(x_i)$.
The second derivative
Rewrite the approximate of the second derivative with


Write Introduction, Thomas algorithm, 




\subparagraph{\textcolor{teal}{Solution:}}
The set of equation for $h^2f_i$ is given by
\begin{equation*}
   -v_{i+1}-v_{i-1}+2v_i = h^2f_i  \hspace{0.5cm} 
\end{equation*}
Rearranging the equation gives
\begin{equation*}
  -v_{i-1}+2v_i - v_{i+1}= h^2f_i  \hspace{0.5cm} 
\end{equation*}
Determining the equations for the boundary conditions, $i=1$ holds 
\begin{equation*}
\begin{aligned}
  -v_{0}+2v_1 - v_{2}&= h^2f_1  \hspace{0.5cm} \\
  0 + 2v_1 - v_{2}&= h^2f_1  \hspace{0.5cm} 
\end{aligned}
\end{equation*}

since $v_0=0$. Given $v_{n+1}=0$, similarly holds for $i=n$
\begin{equation*}
\begin{aligned}
  -v_{n-1}+2vn - v_{n+1}&= h^2f_n  \hspace{0.5cm} \\
 -v_{n-1}+ 2v_n - 0&= h^2f_n  \hspace{0.5cm} 
\end{aligned}
\end{equation*}
A combination of the above equations results in 
\begin{equation*}
\begin{aligned}
  2v_1 - v_{2}&= h^2f_1 & = \tilde{b_1}  \hspace{0.5cm} \\
  ...\\
  -v_{i-1}+2v_i - v_{i+1}&= h^2f_i &= \tilde{b_i}  \hspace{0.5cm} \\
  ...\\
 -v_{n-1}+ 2v_n &= h^2f_n &= \tilde{b_n}  \hspace{0.5cm} 
\end{aligned}
\end{equation*}
Thus, there are $n$ equations that must be computed ($i \in [1,n])$.
The set of equations can be rewritten in matrix-form, there $ \mathbf{A} \in \R^{n \times n} $ and $ \mathbf{v}, \mathbf{b} \in \R^{n}$
\[
     \begin{bmatrix}
                           2& -1 & 0 &\dots   & \dots &\dots \\
                           -1 & 2 & -1 &\dots &\dots &\dots \\
                           & -1 & 2 & 0 & \dots & \dots \\
                           & \dots   & \dots &\dots   &\dots & \dots \\
                           &   &  &-1  &2& -1 \\
                           &    &  &   &-1 & 2 \\
                      \end{bmatrix}\begin{bmatrix}
                           v_1\\
                           v_2\\
                           \dots \\
                          \dots  \\
                          \dots \\
                           v_n\\
                      \end{bmatrix}
  =\begin{bmatrix}
                          \tilde{ b_1}\\
                          \tilde{b_2}\\
                           \dots \\
                           \dots \\
                          \dots \\
                           \tilde{b_n}\\
                      \end{bmatrix}.
\]


\subparagraph{\textcolor{teal}{Solution}}
In the following, the vectors $ \mathbf{a}, \mathbf{b}, \mathbf{c}$ are defined as the matrix-elements, there $ \mathbf{b}$ is placed along the diagonal, $\mathbf{a}$ is the lower diagonal and $\mathbf{c}$ is the upper diagonal. $g_i$ stands for $h^2f_i$, the solution for each equation. \\
\[
    \mathbf{A} = \begin{bmatrix}
                           b_1& c_1 & 0 &\dots   & \dots &\dots \\
                           a_1 & b_2 & c_2 &\dots &\dots &\dots \\
                           & a_2 & b_3 & c_3 & \dots & \dots \\
                           & \dots   & \dots &\dots   &\dots & \dots \\
                           &   &  &a_{n-2}  &b_{n-1}& c_{n-1} \\
                           &    &  &   &a_{n-1} & b_n \\
                      \end{bmatrix}\begin{bmatrix}
                           u_1\\
                           u_2\\
                           \dots \\
                          \dots  \\
                          \dots \\
                           u_n\\
                      \end{bmatrix}
  =\begin{bmatrix}
                           \tilde{g}_1\\
                           \tilde{g}_2\\
                           \dots \\
                           \dots \\
                          \dots \\
                           \tilde{g}_n\\
                      \end{bmatrix}.
\]
A layout of the general algorithm: \\
Read from file or terminal getting the input variables (a,b,c,n) \\
Initialization: n number of integration points \\
Allocation of memory for $ \mathbf{a}, \mathbf{b}, \mathbf{c}, \mathbf{g}$ \\
Forward solution loop: for (i=1, n-1) update $\mathbf{b}, \mathbf{g}$ \\
Backward solution loop: for (i=-2,1) update $\mathbf{b}$ \\
Testing analytical solution: compute relative error $  \epsilon =  \left\lvert \frac{u_i-exact_i}{exact_i} \right\rvert$ \\
Write results to file \
Number of FLOPS for $\mathbf{b}$ \\
%Initialize (specific case): 2 FLOPs x (n-1) \\
Forward loop: (special case): 2 FLOPs x (n-2) \\
Forward loop: (general case): 3 FLOPs x (n-2) \\
Backward loop: 1 FLOP and 3 FLOPs x (n-2) \\

Number of FLOPS for $\mathbf{g}$ \\
Initialize: 4 FLOPs x (n-1) \\
Forward loop: 3 FLOPs x (n-2)\\
Number of FLOPS for $exact$\\
Initialize: 5 FLOPs x (n-1)













% ------------------- end of main content ---------------

\end{document}

