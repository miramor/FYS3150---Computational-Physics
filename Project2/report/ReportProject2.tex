%-------------------- begin preamble ----------------------

\documentclass[%
oneside,                 % oneside: electronic viewing, twoside: printing
final,                   % draft: marks overfull hboxes, figures with paths
10pt]{article}

\listfiles               %  print all files needed to compile this document
\usepackage{url}			% needed for citing
\usepackage{relsize,makeidx,color,setspace,amsmath,amsfonts,amssymb}
\usepackage[table]{xcolor}
\usepackage{bm,ltablex,microtype}
\newcommand{\R}{\mathbb{R}} 

\usepackage[pdftex]{graphicx}

\usepackage{fancyvrb} % packages needed for verbatim environments

\usepackage[T1]{fontenc}
%\usepackage[latin1]{inputenc}
\usepackage{ucs}
\usepackage[utf8x]{inputenc}

\usepackage{lmodern}         % Latin Modern fonts derived from Computer Modern

% Hyperlinks in PDF:
\definecolor{linkcolor}{rgb}{0,0,0.4}
\usepackage{hyperref}
\hypersetup{
    breaklinks=true,
    colorlinks=true,
    linkcolor=linkcolor,
    urlcolor=linkcolor,
    citecolor=black,
    filecolor=black,
    %filecolor=blue,
    pdfmenubar=true,
    pdftoolbar=true,
    bookmarksdepth=3   % Uncomment (and tweak) for PDF bookmarks with more levels than the TOC
    }
%\hyperbaseurl{}   % hyperlinks are relative to this root

\setcounter{tocdepth}{2}  % levels in table of contents


% prevent orhpans and widows
\clubpenalty = 10000
\widowpenalty = 10000

% --- end of standard preamble for documents ---


% insert custom LaTeX commands...

\raggedbottom
\makeindex
\usepackage[totoc]{idxlayout}   % for index in the toc
% \usepackage[nottoc]{tocbibind}  % for references/bibliography in the toc

% \usepackage{booktabs}				% for citing
\usepackage{float}					% for placing figures
%-------------------- end preamble ----------------------

\begin{document}




% matching end for #ifdef PREAMBLE



% ------------------- main content ----------------------



% ----------------- title -------------------------

\thispagestyle{empty}

\begin{titlepage}
   \begin{center}
       \vspace*{1cm}
       \Huge\textbf{----------}
\end{center}
\begin{center}

       \vspace{0.5cm}
       \large
        ------------
            
       \vspace{1.5cm}

       \textbf{Mira Mors and Elias Tidemand Ruud}

       \vfill

 
            
      {\bf \href{{http://www.uio.no/studier/emner/matnat/fys/FYS3150/index-eng.html}}{Computational Physics I FYS3150/FYS4150}} 
            
       \vspace{0.8cm}
     
            
       Department of Physics\\
       University of Oslo\\
       Norway\\
       September 2020
            
   \end{center}
\end{titlepage}





\tableofcontents
\newpage
\section{Abstract}
\section{Introduction}

Since $\mathbf{U}$ is orthogonal holds $\mathbf{U^TU} = I$
\begin{equation}
w_i^Tw_j = (Uv_i)^T(Uv_j)=v_i^TU^TUv_j=v_i^TIv_j = v_i^Tv_j
\end{equation}

\end{document}